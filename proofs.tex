\documentclass{article}[10pt]
\usepackage{geometry}
\geometry{legalpaper, margin=2cm}
\usepackage{amsmath}
\usepackage{amsfonts}
\usepackage{tikz}
\newcommand\concat{\ensuremath{\mathbin{+\mkern-8mu+}}}
\title{Exercises from Introduction to Functional Programming}
\author{Raoul Schaffranek}
\date{September 14, 2018}
\begin{document}
  \maketitle
  \section{Exercise 5.3.4}
    Prove the laws:
    \begin{align*}
      take\ m\ (drop\ n\ xs) &= drop\ n\ (take\ (m + n)\ xs)\\
      drop\ m\ (drop\ n\ xs) &= drop\ (m + n)\ xs
    \end{align*}
    for every natural number $m$ and $n$ and every finite list $xs$.
    \paragraph{Proof $take\ m\ (drop\ n\ xs) = drop\ n\ (take\ (m + n)\ xs)$}
    By induction on $xs$:
    \begin{itemize}
      \item Case $[]$
        \begin{align*}
          &\mathrel{\hphantom{=}} take\ m\ (drop\ n\ [])\\
          &= take\ m\ []               &\text{by } drop.2\\
          &= []                        &\text{by } take.2\\
          &= take\ (m+n)\ []           &\text{by } take.2\\
          &= drop\ n\ (take (m+n)\ []) &\text{by } drop.2
        \end{align*}
      \item Case $(x:xs)$. By induction on $n$:
        \begin{itemize}
          \item Case $0$
          \begin{align*}
            &\mathrel{\hphantom{=}} take\ m\ (drop\ 0\ (x:xs))\\
            &= take\ m\ (x:xs)                &\text{by }drop.1\\
            &= take\ (m+0)\ (x:xs)            &0 \text{ right-neutral w.r.t }+\\
            &= drop\ 0\ (take\ (m+0)\ (x:xs)) &\text{by }drop.1
          \end{align*}
          \item Case $(n+1)$
          \begin{align*}
            &\mathrel{\hphantom{=}} take\ m\ (drop\ (n+1)\ (x:xs))\\
            &= drop\ (n+1)\ (take\ (m+(n+1))\ (x:xs))          &\text{by I.H.}
          \end{align*}
        \end{itemize}
    \end{itemize}
    \paragraph{Proof $drop\ m\ (drop\ n\ xs) = drop\ (m+n)\ xs$}
    By induction on $xs$:
    \begin{itemize}
      \item Case $[]$:
      \begin{align*}
        &\mathrel{\hphantom{=}} drop\ m\ (drop\ n\ [])\\
        &= drop\ m\ []                         &\text{by }drop.2\\
        &= []                                  &\text{by }drop.2\\
        &= drop (m+n) []                       &\text{by }drop.2
      \end{align*}
      \item Case $(x:xs)$, by induction on $n$:
      \begin{itemize}
        \item Case $0$:
          \begin{align*}
            &\mathrel{\hphantom{=}} drop\ m\ (drop\ 0\ (x:xs))\\
            &=drop\ m\ (x:xs)            &\text{by }drop.1\\
            &=drop\ (m+0)\ (x:xs)        &0 \text{ right-neutral w.r.t. }+
          \end{align*}
        \item Case $(n+1)$
          \begin{align*}
            &\mathrel{\hphantom{=}} drop\ m\ (drop\ (n+1)\ (x:xs))\\
            &=drop\ (m+(n+1))\ (x:xs) &\text{by I.H.}\\
          \end{align*}
      \end{itemize}
    \end{itemize}
  \section{Exercise 5.3.5}
    Prove the laws:
    \begin{align*}
      map\ (f . g)\ xs      &= map\ f\ (map\ g\ xs)\\
      map\ f\ (concat\ xss) &= concat\ (map\ (map\ f)\ xss)
    \end{align*}
    for every function $f$ and $g$, finite list $xs$, and finite list of finite lists $xss$.
    \paragraph{Proof $map\ (f . g)\ xs = map\ f\ (map\ g\ xs)$}
    By induction on $xs$:
    \begin{itemize}
      \item Case $[]$:
      \begin{align*}
        map\ (f . g)\ [] &= []                   &\text{ by definition of }map\\
                         &= map\ g\ []           &\text{ by definition of }map\\
                         &= map\ f\ (map\ g\ []) &\text{ by definition of }map
      \end{align*}
      \item Case $(x:xs)$:
      \begin{align*}
        map\ (f . g)\ (x:xs) &= (f . g)(x) : map\ (f . g)\ xs    &\text{by definition of }map\\
                             &= f(g(x)) : map\ (f . g)\ xs       &\text{by definition of }.\\
                             &= f(g(x)) : (map\ f\ (map\ g\ xs)) &\text{by I.H}\\
                             &= map\ f\ (g(x) : (map\ g\ xs))    &\text{by definition of }map\\
                             &= map\ f\ (map\ g\ (x:xs))         &\text{by definition of }map
      \end{align*}
    \end{itemize}
  \section{Exercise 7.2.1}
    What is the value of:
    \begin{equation*}
      map\ (3 \times)\ [0 .. ] = iterate\ (+3)\ 0
    \end{equation*}
    when '=' means denotational equality? What is its value when '=' means
    computable equality?
    \paragraph{Solution}
    In the case of denotational equality, we can prove that the claim holds, i.e. evaluates to $True$.
    Since equality is chain-complete, we use the take lemma and show that the claim holds for all
    natural numbers. However, a naive structural induction will fail, because the induction hypothesis
    will not be applicable directly. To see this, let's start with a naive approach:
    \begin{align*}
      map\ (3 \times)\ [0 .. ] &= iterate\ (+3)\ 0 \text{ iff }\\
      take\ n\ (map\ (3 \times)\ [0..]) &= take\ n\ (iterate\ (+3)\ 0) &\text{by take-lemma}
    \end{align*}
    By induction on $n$:
    \begin{itemize}
      \item Case $0$:
      \begin{align*}
        &\mathbin{\hphantom{\equiv}}take\ 0\ (map\ (3 \times)\ [0..]) = take\ 0\ (iterate\ (+3)\ 0)\\
        &\equiv []                                                    = take\ 0\ (iterate\ (+3)\ 0) &\text{ by }(take.1)\\
        &\equiv []                                                    = []                          &\text{ by }(take.1)\\
        &\equiv True &\text{ by }(=)\\
      \end{align*}
      \item Case $n+1$
      \begin{align*}
        &\mathbin{\hphantom{\equiv}}take\ (n+1)\ (map\ (3 \times)\ [0..]) = take\ (n+1)\ (iterate\ (+3)\ 0)\\
        &\equiv take\ (n+1) (((3 \times)\ 0) : (map\ (3 \times)\ [1..]) =  take\ (n+1)\ (iterate\ (+3)\ 0) &\text{ by }(map)\\
        &\equiv ((3 \times)\ 0) : (take\ n (map\ (3 \times)\ [1..])) = take\ (n+1)\ (iterate\ (+3)\ 0) &\text{ by }(take.3)\\
        &\equiv ((3 \times)\ 0) : (take\ n (map\ (3 \times)\ [1..])) = take\ (n+1)\ (0 : (iterate\ (+3)\ ((+3)\ 0))) &\text{ by }(iterate)\\
        &\equiv ((3 \times)\ 0) : (take\ n (map\ (3 \times)ß [1..])) = 0 : (take\ n\ (iterate\ (+3)\ ((+3)\ 0))) &\text{ by }(take.3)\\
        &\equiv 0 : (take\ n\ (map\ (3 \times)\ [1..])) = 0 : (take\ n\ (iterate\ (+3)\ ((+3)\ 0))) &\text{ by }(\times)\\
        &\equiv 0 : (take\ n\ (map\ (3 \times)\ [1..])) = 0 : (take\ n\ (iterate\ (+3)\ 3)) &\text{ by }(+)\\
        &\equiv (take\ n\ (map\ (3 \times)\ [1..])) = (take\ n\ (iterate\ (+3)\ 3)) &\text{ by }(=)\\
        &\mathbin{\hphantom{\equiv}} \text{We are now stuck, becuase the I.H. is not applicable}
      \end{align*}
    \end{itemize}
    For this reason, we show an even stronger property, namely that
    the claim holds, if $0$ is replaced by $m$ on on the left-hand side of the equation and by $3 \times m$
    on the right hand side of the equation. The original claim then follows from the fact that $0 = 3 \times 0$.
    \begin{align*}
      map\ (3 \times)\ [0 .. ] &= iterate\ (+3)\ 0 \text{ if }\\
      map\ (3 \times)\ [m..] &= iterate\ (+3)\ (3*m) \text{ iff }\\
      take\ n\ (map\ (3 \times)\ [m..]) &= take\ n\ (iterate\ (+3)\ (3 \times m)) &\text{by take-lemma}
    \end{align*}
    By induction on $n$:
    \begin{itemize}
      \item Case $0$:
      \begin{align*}
        &\mathbin{\hphantom{\equiv}}take\ 0\ (map\ (3 \times)\ [m..]) = take\ 0\ (iterate\ (+3)\ (3 \times m))\\
        &\equiv []                                                    = take\ 0\ (iterate\ (+3)\ (3 \times m)) &\text{ by }(take.1)\\
        &\equiv []                                                    = []                          &\text{ by }(take.1)\\
        &\equiv True &\text{ by }(=)\\
      \end{align*}
      \item Case $n+1$
      \begin{align*}
        &\mathbin{\hphantom{\equiv}}take\ (n+1)\ (map\ (3 \times)\ [m..]) = take\ (n+1)\ (iterate\ (+3)\ (3 \times m))\\
        &\equiv take\ (n+1) (((3 \times)\ m) : (map\ (3 \times) [(m+1)..]) =  take\ (n+1)\ (iterate\ (+3)\ (3 \times m)) &\text{ by }(map)\\
        &\equiv ((3 \times)\ m)) : (take\ n\ (map\ (3 \times) [(m+1)..])) = take\ (n+1)\ (iterate\ (+3)\ (3 \times m)) &\text{ by }(take.1)\\
        &\equiv (3 \times m) : (take\ n\ (map\ (3 \times) [(m+1)..])) = take\ (n+1)\ (iterate\ (+3)\ (3 \times m)) &\text{ by }(application)\\
        &\equiv (3 \times m) : (take\ n\ (map\ (3 \times) [(m+1)..])) = (take\ (n+1) ((3 \times m) : iterate\ (+3)\ ((+3)\ (3 \times m)))) &\text{ by }(iterate)\\
        &\equiv (3 \times m) : (take\ n\ (map\ (3 \times) [(m+1)..])) = (3 \times m) : (take\ n (iterate\ (+3)\ ((+3)\ (3 \times m)))) &\text{ by }(take)\\
        &\equiv take\ n\ (map\ (3 \times) [(m+1)..]) = take\ n\ (iterate\ (+3)\ ((+3)\ (3 \times m))) &\text{ by }(=)\\
        &\equiv take\ n\ (map\ (3 \times) [(m+1)..]) = take\ n\ (iterate\ (+3)\ (3 \times (m+1))))\\
        &\mathbin{\hphantom{=}}\text{by } \text{ distributivity of } \times \text{ and } +\\
        &\equiv True &\text{by I.H.}
      \end{align*}
    \end{itemize}
    In the case of computational equality, however, the value will be $\bot$. We show, that the left hand side of the
    equation suffers from case-exhaustion.
    \begin{align*}
      map\ (3 \times)\ [0 .. ]
    \end{align*}
  \section{Exercise 7.5.4}
    In Chapter 5 a proof was given that:
    \begin{equation*}
      take\ n\ xs \concat drop\ n\ xs = xs
    \end{equation*}
    for all finite lists xs. Extend the proof to cover infinite lists xs. 
    \paragraph{Proof} First, notice that equality is chain-complete. It therefore suffices
    to show that the claim holds for all partial lists $xs$.
    \begin{itemize}
      \item Case $\bot$, by induction on $n$:
      \begin{itemize}
        \item Case $0$:
        \begin{align*}
          &\mathbin{\hphantom{=}}take\ 0\ \bot \concat drop\ 0\ \bot\\
          &=                                [] \concat drop\ 0\ \bot &\text{ by }(take.1)\\
          &=                                           drop\ 0\ \bot &\text{ by }(\concat)\\
          &=                                           \bot          &\text{ by }(drop.0)
        \end{align*}
        \item $Case (n+1)$:
        \begin{align*}
          &\mathbin{\hphantom{=}}take\ (n+1)\ \bot \concat drop\ (n+1)\ \bot\\
          &= \bot                                  \concat drop\ (n+1)\ \bot &\text{ by }(take.0)\\
          &= \bot                                  \concat \bot              &\text{ by }(drop.0)\\
          &= \bot                                                            &\text{ by }(\concat)
        \end{align*}
      \end{itemize}
      \item Case $(x:xs)$ where $xs$ is partial, by induction on $n$:
      \begin{itemize}
        \item Case $0$:
        \begin{align*}
          &\mathbin{\hphantom{=}}take\ 0\ (x:xs) \concat drop\ 0\ (x:xs)\\
          &=                     []              \concat drop\ 0\ (x:xs) &\text{ by }(take.1)\\
          &=                                             drop\ 0\ (x:xs) &\text{ by }(\concat)\\
          &=                                             (x:xs)          &\text{ by }(drop.1)
        \end{align*}
        \item Case $(n+1)$:
        \begin{align*}
          &\mathbin{\hphantom{=}}take\ (n+1)\ (x:xs) \concat drop\ (n+1)\ (x:xs)\\
          &= x : take\ n\ xs \concat drop\ (n+1)\ (x:xs) &\text{by }(take.2)\\
          &= x : take\ n\ xs \concat drop\ n\ xs         &\text{by }(drop.2)\\
          &= x : (take\ n\ xs \concat drop\ n\ xs)     &: \text{ associative}\\
          &= x : xs                                    &\text{by I.H}
        \end{align*}
      \end{itemize}
    \end{itemize}
\end{document}
